\begin{center}
	\textbf{\Large{Abstract}}
\end{center}

\vspace{0.5cm}

\justifying

The present project aims to develop a numerical experiment to understand the impact of important parameters in the glacier dynamics, such as the grounding line. Using the finite element method Elmer/Ice, simulations on idealized topographies are proposed, which are set up in the context of the CalvingMIP inter-comparison project, that aims to develop different models to simulate and to improve calving laws in ice sheet models. Using these idealized topographies, the numerical experiments are performed using resolutions varying from 10km until 1km, starting from an initial state where there is no ice, until the formation of the glacier. The objective of the study is to evaluate the impact of the resolution on the prediction of the grounding line position after the system has reached the steady state. The analysis of the results are contrasted with the theory and show the convergence of the behaviour of the grounding line position for higher resolutions, where the differences start to be lower.

\vspace{3cm}

\clearpage